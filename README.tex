\documentclass{article}

\usepackage{amsfonts, amsmath, fontspec, minted, stmaryrd, url}

\setmonofont{Symbola}
\setmonofont{DejaVu Sans Mono}

\title{coq-alpha-pearl README}
\author{Joshua Grosso \\ Faculty Mentor: Michael Vanier}

\begin{document}

\maketitle

\section{Introduction}

A Coq formalization of ``Functional Pearls: α-conversion is easy'' (Altenkirch, 2002). The most important file is [theories/Alpha.v].

To our knowledge, all of the main results contained within the paper itself have been formalized. (Proving the equivalence to de Bruijn terms is still in progress, but because it is left to the reader in the original paper, we are comfortable sharing this formalization as-is.)

\section{Technical Details}

\subsection{Project Setup}

Install the \verb|math-comp| and \verb|extructures| libraries, and then run \verb|make html| to
build the project and its documentation (which annotates important definitions with their
counterparts in the original paper).

\subsection{Project Structure}

The project is organized on disk as follows:

\verb|theories/Alpha.v| contains the material from the original paper. It is parameterized by the infinite set $\mathcal{V}$ and its corresponding $\textrm{Fresh}$ function.

\verb|theories/AlphaImpl.v| contains two concrete implementations of the formalization, one where
$\mathcal{V}$ is the set of finite strings and another where $\mathcal{V} = \mathbb{N}$. We expect
the former to be the default choice for practical purposes; we implemented the latter merely as a
curiosity.

\verb|theories/Examples.v| contains several examples of $\equiv_\alpha$ in action.

\verb|theories/Util.v| and \verb|theories/Util/*.v| provide a small, custom utility library for the
project. The one exception is \verb|theories/Util/PlfTactics.v|, which is a copy of
\verb|LibTactics.v| from \emph{Programming Language Foundations}.

\subsection{External Libraries}

The project relies on the \verb|math-comp| library as a supplement to Coq's built-in standard
library. It relies on \verb|extructures| for extensionally-equal finite sets and maps. Finally, it
relies on \verb|LibTactics.v| from \emph{Programming Language Foundations} (it exists as a copy in our
project as \verb|theories/Util/PlfTactics.v|).

\section{Formalization Overview}

This formalization contains several deviations from the original paper. Some were for ease of
formalization, while others were (as best we can tell) required to prove the desired statements.
We tried to be generally faithful to the provided proofs, at least for the more complex theorems.
Finally, we admit the possibility that any or all of the below changes are
misinterpretations on our part of the original author's meaning.

\subsection{Typed Lambda Calculus}

Our representation of the lambda calculus is untyped, for convenience. However, because the material
is agnostic to the presence of types, it should be trivial to adapt it for e.g.\ the simply-typed
$\lambda$-calculus.

\subsection{(In)finite Sets}

We encode the infinite set $\mathcal{V}$ as a type with an order:

\begin{minted}{coq}
Parameter V : ordType.
\end{minted}

(Note that defining $\mathcal{V}$ as an \verb|ordType| provides decidable equality).

The ordering constraint is not strictly necessary for formalization (and is not present in the
original paper), but it allows us to use the \verb|extructures| library's implementation of
extensionally-equal finite sets. This is very convenient for formalization, and we expect
that orderings exist for all usual definitions of $\mathcal{V}$ (e.g.\ the set of all finite
strings, or $\mathbb{N}$). Of course, if that expectation proves to be incorrect, we could remove
this constraint (for example, we could represent finite sets as linked lists with an appropriate
equivalence relation, which we would manually use in lieu of definitional equality).

\subsection{Finite Maps}

We define partial bijections via the \verb|extructures| library's implementation of
extensionally-equal finite maps. We define substitutions similarly; we did not directly encode them
as functions because we found that finite maps were more convenient (given that we are avoiding
the use of classical logic).

\subsection{Inductive Definitions as Boolean Predicates}

Where possible, we preferred boolean predicates to inductive propositions (being sure, however, to prove
equivalence with the original definitions). For example,
consider the definition of $\textrm{Tm}(X)$ (page 2). A straightforward formalization uses
an inductive proposition, like so:

\begin{minted}{coq}
(* [x ∈ X] is notation for SSReflect's [x \in X]. *)
Section in_Tm.
  #[local] Reserved Notation "t '∈' 'Tm' X" (at level 40).

  Inductive in_Tm : {fset V} -> term -> Prop :=
  | Tm_variable : forall X x,
    x \in X ->
    variable x ∈ Tm X
  | Tm_application : forall X t u,
    t \in Tm X -> u \in Tm X ->
    application t u \in Tm X
  | Tm_abstraction : forall X t x,
    t \in Tm (X ∪ {x}) ->
    abstraction x t \in Tm X

  where "t '∈' 'Tm' X" := (in_Tm X t).
End in_Tm.
\end{minted}

However, we can alternately define $\textrm{Tm}$ as a boolean predicate (assuming suitable
definitions of $\textrm{FV}$ and $\subseteq$):

\begin{minted}{coq}
Definition Tm X t : bool := FV t ⊆ X.
\end{minted}

Although it is admittedly subjective, we find the latter definition to be semantically clearer. We
have proven that these two representations are equivalent:

\begin{minted}{coq}
Lemma TmP : forall X t, reflect (in_Tm X t) (t ∈ Tm X).
\end{minted}

Similarly, we define, use, and prove correctness of algorithmic forms of $\equiv_\alpha^R$ and
$\textrm{Tm}^{\textrm{db}}$.

\subsection{Explicitly-Provided Sets}

When the original paper introduces a term $t \in \textrm{Tm}(X)$, we note a slight ambiguity: Given
that $\forall X, Y : X \subseteq Y \implies \textrm{Tm}(X) \subseteq \textrm{Tm}(Y)$, it is possible
that $\textrm{FV}(t) \subset X$ and thus $X$ is not uniquely specified. For ease of formalization
(and, to our knowledge, without loss of generality), we have largely replaced references to these
sets with $\textrm{FV}(t)$. For example, we have formalized Proposition 2.1 as $\forall t : t
\equiv_\alpha^{1_{\textrm{FV}(t)}} t$ instead of $\forall t \in \textrm{Tm}(X) : t
\equiv_\alpha^{1_X} t$. For another example, we defined $t \equiv_\alpha u = t
\equiv_\alpha^{1_{\textrm{FV}(t)}} u$ (noting that $\textrm{FV}(t) = \textrm{FV}(u)$ by the assumed
$\alpha$-equivalence of $t$ and $u$).

We employ a similar strategy for function domains and codomains. Instead of referencing the set $X$
in $f \in X \longrightarrow \textrm{Tm}(Y)$ (page 4), we instead reference $\textrm{dom}(f)$.
Similarly, instead of referencing $Y$, we instead calculate the smallest such set by iterating over
the codomain of $f$ (see \verb|codomm_Tm_set|).

\subsection{Symmetric Updates}

Assuming the original paper uses ordered tuples, we have changed the definition of $R(x,y) = (R
\setminus \{ (x,v),(y,w) | v,w \in \mathcal{V} \}) \cup \{ (x,y) \}$ to $R(x,y) = (R \setminus \{
(x,v),(w,y) | v,w \in \mathcal{V} \}) \cup \{ (x,y) \}$. We believe this to have been the intention
of the original author (given the emphasis on the symmetricality of the update procedure).

\subsection{Lemma 1.3}

We believe that the original proof for Lemma 1.3 (page 3) has a counterexample, arising from the second
and third steps of the $\iff$ chain. Specifically, consider the backwards implication portion of
that step. Let us separately consider the two cases of our hypothesis $(a = x \land z = b) \lor (x
\neq a \land z \neq b \land a R; S b)$. If $a = x$ and $z = b$, then we can let $c = y$ and our
conclusion is satisfied. Thus, we will now assume instead that $x \neq a$, $z \neq b$, and $a R; S b$.
Because $R$ and $S$ are both partial bijections, there must exist a unique $c$ for which $aRc$ and
$cSb$. However, as best we can tell, $c = y$ is not prohibited by the assumptions of the lemma;
thus, for the sake of argument, we will assume this equality. In this case, neither case of the
conclusion is possible, and so we have a contradiction.

In our formalization, we have replaced $R(x,y); S(y,z) = (R;S)(x,z)$ with $\forall a, b \in
\mathcal{V} : a R(x,y); S(y,z) b \implies a (R;S)(x,z) b$. Fortunately, this is still strong enough to
prove proposition 2.3, i.e.\ where lemma 1.3 was originally used.

\subsection{Equivalence Classes}

Coq does not yet support quotient types (at least not without additional axioms, which we were
hesitant to add). Thus, instead of defining $Tm(X)/\equiv_\alpha$ at the term-level, we manually
proved that the subsequent theorems respect $\equiv_\alpha$.

\subsection{Substitution Extension}

In the definition of $\llparenthesis f \rrparenthesis$ (page 4), we were unclear whether or not $Y$
is updated upon structural recursion, i.e.\ whether we should always reference the codomain of the
top-level $f$, or if we should instead use $\textrm{cod}(f[x,z])$ when recursing in $\llparenthesis
f \rrparenthesis(\lambda x. t) = \lambda z. \llparenthesis f[x,z] \rrparenthesis (t)$. We assumed
the latter, and were able to prove the subsequent theorems.

\subsection{Substitution Monad Laws}

We found it necessary to add an additional condition $x \notin \textrm{FV}(v)$ to the second
monad law for substitution, a la Exercise 2.2 in
\url{http://www.cse.chalmers.se/research/group/logic/TypesSS05/Extra/geuvers.pdf}.

\subsection{Lemma 7}

We instead proved $\llparenthesis f \rrparenthesis (t) \equiv_\alpha^{f^\circ} t$; we assume this
was the original author's intention.

\subsection{Minor Textual Notes}

On page 2, we presume that $X \subseteq_{\textrm{fin}} \mathcal{V}$ is equivalent (by the infiniteness of
$\mathcal{V}$) to $X \subset_{\textrm{fin}} \mathcal{V}$.

On page 3, we presume that $R(x,y) \dots \in (X \cup \{ x \}) \times (Y \cup \{ y \})$ uses $\in$ to
represent $\subseteq$.

On page 5, we assume that ``Hence by ind.hyp.\ we know $\llparenthesis f[x,z] \rrparenthesis (t) \equiv_\alpha^{S(z,z')}
\llparenthesis f[y,z'] \rrparenthesis (u)$\dots'' was intended to read ``Hence by ind.hyp.\ we know $\llparenthesis f[x,z] \rrparenthesis (t) \equiv_\alpha^{S(z,z')}
\llparenthesis g[y,z'] \rrparenthesis (u)$\dots''.

\end{document}