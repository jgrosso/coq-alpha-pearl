\documentclass{article}

\usepackage{amsfonts, amsmath, fontspec, hyperref, minted, stmaryrd}

\setmonofont[Scale=0.9]{DejaVu Sans Mono}

\title{coq-alpha-pearl README}
\author{Joshua Grosso \\ Faculty Mentor: Michael Vanier}

\begin{document}

\maketitle

\section{Introduction}

This is a Coq formalization of ``Functional Pearls: $\alpha$-conversion is easy'' (Altenkirch,
2002). The most important file is \verb|theories/Alpha.v|.

The original paper presents an elegant definition of $\alpha$-equivalence, along with related
algorithms and theorems. By formalizing this material, we present a machine-checkable and
-executable version of this work. We hope that this library exposes definitions of
$\alpha$-equivalence, substitution, etc.\ that are both easy to use computationally and also easy to
reason about in proofs.

To our knowledge, all of the main results contained within the paper itself have been formalized.
(Proving equivalence to de Bruijn terms is still in progress, but because it is left to the reader
in the original paper, we are comfortable sharing this formalization as-is.)

\section{Technical Details}

\subsection{Project Setup}

Install the \verb|mathcomp-ssreflect| and \verb|extructures| libraries (see below), and then run
\verb|make html| to build the project and its documentation (which annotates important definitions
with their counterparts in the original paper).

Detailed installation instructions are as follows: First, install
\href{https://opam.ocaml.org/doc/Install.html}{OPAM}. Then, at the command line, add the Coq
repository to OPAM via \verb|opam repo add coq-released https://coq.inria.fr/opam/released|.
Finally, install the project's dependencies via
\verb|opam install coq-mathcomp-ssreflect coq-deriving coq-extructures|. Now, you should be able to build the
project and its documentation via \verb|make html|. (At the moment, \verb|theories/Alpha.v| and
\verb|theories/Examples.v| may take some time to compile; this is expected.)

\subsection{Project Structure}

The project is organized on disk as follows:

\verb|theories/AlphaPaper.v| contains the material from the original paper. It is parameterized by the
infinite set $\mathcal{V}$ and its corresponding $\textrm{Fresh}$ function.

\verb|theories/AlphaImpl.v| contains two concrete implementations of the formalization, one where
$\mathcal{V}$ is the set of finite strings and another where $\mathcal{V} = \mathbb{N}$. We expect
the former to be the default choice for practical purposes; we implemented the latter merely as a
proof-of-concept.

\verb|theories/Examples.v| contains several examples of $\equiv_\alpha$ in action.

\verb|theories/Util.v| and \verb|theories/Util/*.v| provide a small, custom utility library for the
project. Note that \verb|theories/Util/PlfTactics.v| is a copy of \verb|LibTactics.v| from
\href{https://softwarefoundations.cis.upenn.edu/plf-current/}{\emph{Programming Language
Foundations}} (Pierce et al., 2021).

\subsection{External Libraries}

The project relies on the \href{https://math-comp.github.io}{Mathematical Components} library as a
supplement to Coq's built-in standard library. We also rely on
\href{https://github.com/arthuraa/extructures}{extructures} for extensionally-equal finite sets and
maps. Finally, we rely on \verb|LibTactics.v| from \emph{Programming Language Foundations}.

\section{Formalization Overview}

This formalization contains several deviations from the original paper. Some were for ease of
formalization, while others were (as best we can tell) required to prove the desired statements. We
tried to be generally faithful to the provided proofs, at least for the more complex theorems.
Finally, although we hope it is not the case, any or all of the changes below could have arisen from
misinterpretations on our part of the original author's meaning.

\subsection{Typed Lambda Calculus}

Our representation of the lambda calculus is untyped, for convenience. However, because the material
is agnostic to the presence of types, it should be trivial to adapt it for e.g.\ the simply-typed
$\lambda$-calculus.

\subsection{(In)finite Sets}

We encode the infinite set $\mathcal{V}$ as a type with an order:

\begin{minted}{coq}
Parameter V : ordType.
\end{minted}

(Note that defining $\mathcal{V}$ as an \verb|ordType| provides decidable equality as well).

The ordering constraint is not strictly necessary for formalization (and is not present in the
original paper), but it allows us to use the \verb|extructures| library's implementation of
extensionally-equal finite sets. This is very convenient for formalization, and we expect that
orderings exist for all the usual definitions of $\mathcal{V}$ (e.g.\ the set of all finite strings,
or $\mathbb{N}$).

Of course, if this expectation proves to be incorrect, we can remove this constraint (at the cost of
simplicity). For example, we could represent finite sets as linked lists with an appropriate
equivalence relation, which we would manually use in lieu of definitional equality.

\subsection{Finite Maps}

We define partial bijections via the \verb|extructures| library's implementation of
extensionally-equal finite maps. We define substitutions similarly; we did not directly encode them
as functions because we found that finite maps were more convenient (given that we are avoiding the
use of classical logic).

\subsection{Inductive Definitions as Boolean Predicates}

Where possible, we preferred boolean predicates to inductive propositions (being sure, however, to
prove equivalence with the original definitions). For example, consider the definition of
$\textrm{Tm}(X)$ (page 2). A straightforward formalization uses an inductive proposition, like so:

\begin{minted}{coq}
(* [x ∈ X] is notation for SSReflect's [x \in X]. *)
Section in_Tm.
  #[local] Reserved Notation "t '∈' 'Tm' X" (at level 40).

  Inductive in_Tm : {fset V} → term → ℙ :=
  | Tm_variable : ∀ X x,
    x ∈ X →
    variable x ∈ Tm X
  | Tm_application : ∀ X t u,
    t ∈ Tm X → u ∈ Tm X →
    application t u ∈ Tm X
  | Tm_abstraction : ∀ X t x,
    t ∈ Tm (X ∪ {x}) →
    abstraction x t ∈ Tm X

  where "t '∈' 'Tm' X" := (in_Tm X t).
End in_Tm.
\end{minted}

However, we can alternately define $\textrm{Tm}$ as a boolean predicate (assuming suitable
definitions of $\textrm{FV}$ and $\subseteq$):

\begin{minted}{coq}
Definition Tm X t : bool := FV t ⊆ X.
\end{minted}

Although it is admittedly subjective, we find the latter definition to be semantically clearer. We
have proven that these two representations are equivalent:

\begin{minted}{coq}
Lemma TmP : ∀ X t, reflect (in_Tm X t) (t ∈ Tm X).
\end{minted}

Similarly, we define, use, and prove correctness of algorithmic forms of $\equiv_\alpha^R$ and
$\textrm{Tm}^{\textrm{db}}$. This makes them easy to use in runtime computations.

\subsection{Explicitly-Provided Sets}

When the original paper introduces a term by writing $t \in \textrm{Tm}(X)$, we note a slight
ambiguity: Given that $\forall X, Y : X \subseteq Y \implies \textrm{Tm}(X) \subseteq
\textrm{Tm}(Y)$, $X$ is not uniquely specified because any $X \supseteq \textrm{FV}(t)$ would
satisfy the constraint. For ease of formalization (and, to our knowledge, without loss of
generality), we have largely replaced references to sets used in this manner with $\textrm{FV}(t)$.
For example, we have formalized Proposition 2.1 as $\forall t : t \equiv_\alpha^{1_{\textrm{FV}(t)}}
t$ instead of $\forall t \in \textrm{Tm}(X) : t \equiv_\alpha^{1_X} t$. For another example, we
defined $t \equiv_\alpha u = t \equiv_\alpha^{1_{\textrm{FV}(t)}} u$ (noting that $\textrm{FV}(t) =
\textrm{FV}(u)$ by the assumed $\alpha$-equivalence of $t$ and $u$; see
\verb|α_equivalent'_implies_α_equivalent| for a proof).

We employ a similar strategy for function domains and codomains. Instead of referencing the set $X$
in $f \in X \longrightarrow \textrm{Tm}(Y)$ (page 4), we instead reference $\textrm{dom}(f)$.
Similarly, instead of referencing $Y$, we calculate the smallest such set by iterating over the
codomain of $f$ (see \verb|codomm_Tm_set|).

Our original approach had been to name these sets explicitly and take them as parameters to the
propositions. We were successful in proving the main results of the paper, but the resulting
definitions were very unwieldy. Eventually, we revised our work to automatically calculate all such
sets where possible, and found that the proof scripts became much shorter as a result.

\subsection{Symmetric Updates}

(We assumed the original paper uses ordered tuples; if that is not the case, this change is of no
importance.) We changed the definition of $R(x,y) = (R \setminus \{ (x,v),(y,w) \, | \, v,w \in
\mathcal{V} \}) \cup \{ (x,y) \}$ to $R(x,y) = (R \setminus \{ (x,v),(w,y) \, | \, v,w \in
\mathcal{V} \}) \cup \{ (x,y) \}$. We believe this to have been the intention of the original author
(given the emphasis on the symmetricality of the update procedure).

\subsection{Lemma 1.3}

We believe that the original proof for Lemma 1.3 (page 3) has a counterexample, arising from the
second and third steps of the $\iff$ chain. Specifically, consider the backwards implication portion
of $\exists c. (a = x \land y = c \land z = b) \lor (x \neq a \land y \neq c \land z \neq b \land a
R c \land c S b) \iff (a = x \land z = b) \lor (x \neq a \land z \neq b \land a R; S b)$. Let us
separately consider the two cases of our hypothesis $(a = x \land z = b) \lor (x \neq a \land z \neq
b \land a R; S b)$. If $a = x$ and $z = b$, then we can let $c = y$ and our conclusion is satisfied.
Thus, we will now assume instead that $x \neq a$, $z \neq b$, and $a R; S b$. Because $R$ and $S$
are both partial bijections, there must exist a unique $c$ for which $aRc$ and $cSb$. However, as
best we can tell, $c = y$ is not prohibited by the assumptions of the lemma; thus, for the sake of
argument, we will assume this equality. In this case, neither case of the conclusion is possible,
and so we have a contradiction.

In our formalization, we have replaced $R(x,y); S(y,z) = (R;S)(x,z)$ with $\forall a, b \in
\mathcal{V} : a R(x,y); S(y,z) b \implies a (R;S)(x,z) b$. Fortunately, this is still strong enough
to prove proposition 2.3, i.e.\ where lemma 1.3 was originally used.

\subsection{Equivalence Classes}

Coq does not yet support quotient types (at least not without adding related axioms, which we are
hesitant to do). Thus, instead of defining $\textrm{Tm}(X)/\equiv_\alpha$ at the term-level, we
manually proved that the subsequent theorems respect $\equiv_\alpha$.

\subsection{Substitution Extension}

In the definition of $\llparenthesis f \rrparenthesis$ (page 4), we were unclear whether or not $Y$
is held constant during structural recursion, i.e.\ whether we should always reference the codomain
of the top-level $f$, or if we should instead use $\textrm{cod}(f[x,z])$ when recursing in
$\llparenthesis f \rrparenthesis(\lambda x. t) = \lambda z. \llparenthesis f[x,z] \rrparenthesis
(t)$. We assumed the latter, and were able to prove the subsequent theorems.

\subsection{Substitution Monad Laws}

We found it necessary to add an additional condition $x \notin \textrm{FV}(v)$ to the second monad
law for substitution, a la Exercise 2.2 in \emph{Introduction to Lambda Calculus} (Barendregt \&
Barendsen, 2000).

\subsection{Lemma 7}

We instead proved $\llparenthesis f \rrparenthesis (t) \equiv_\alpha^{f^\circ} t$; we assume this
was the original author's intention.

\subsection{Minor Textual Notes}

On page 2, we presume that $X \subseteq_{\textrm{fin}} \mathcal{V}$ is equivalent (by the
infiniteness of $\mathcal{V}$) to $X \subset_{\textrm{fin}} \mathcal{V}$.

On page 3, we presume that $R(x,y) \textellipsis \in (X \cup \{ x \}) \times (Y \cup \{ y \})$ uses $\in$ to
represent $\subseteq$.

On page 5, we assume that ``Hence by ind.hyp.\ we know $\llparenthesis f[x,z] \rrparenthesis (t)
\equiv_\alpha^{S(z,z')} \llparenthesis f[y,z'] \rrparenthesis (u)$\dots'' was intended to read
``Hence by ind.hyp.\ we know $\llparenthesis f[x,z] \rrparenthesis (t) \equiv_\alpha^{S(z,z')}
\llparenthesis g[y,z'] \rrparenthesis (u)$\dots''.

\section{Future Work}

Some of our potential next steps:

Currently, the definition of \verb|term| is not user-extensible (and is currently rather
bare-bones). We hope to explore ways, if possible, to adapt this formalization to more featureful
$\lambda$-calculi on demand.

We hope to investigate and improve the runtime performance of our definition of $\equiv_\alpha$.
(Currently, \verb|theories/Examples.v| is rather slow to run.)

Please see the header comment in \verb|theories/Alpha.v| for tasks relating to the Coq scripts
themselves.

\section{Acknowledgements}

This work was financially supported by the \emph{Soli Deo Gloria} SURF endowment at Caltech.

\end{document}